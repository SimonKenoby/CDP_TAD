\documentclass[a4paper, 11pt, oneside]{article}

\usepackage[utf8]{inputenc}
\usepackage[T1]{fontenc}
\usepackage[francais]{babel}
\usepackage{array}
\usepackage{shortvrb}
\usepackage{listings}
\usepackage[fleqn]{amsmath}
\usepackage{amsfonts}
\usepackage{fullpage}
\usepackage{enumerate}
\usepackage{graphicx}             % import, scale, and rotate graphics
\usepackage{subfigure}            % group figures
\usepackage{alltt}
\usepackage{url}
\usepackage{indentfirst}
\usepackage{eurosym}
\usepackage{listings}
\usepackage{color}

\definecolor{mymauve}{rgb}{0.58,0,0.82}

\begin{document}

\lstset{language=C, commentstyle={\color{blue}}, frame=single,
stringstyle=\color{magenta}}

\title{INFO0947: TAD}
\author{Groupe XX: Simon Lorent, Corentin Jemine}
\date{Avril/Mai 2015}

\maketitle
\clearpage

\section{Type abstrait}
	\subsection{Signature}
	\noindent \textbf{Type:}
	\\ \indent Multi \footnote{Multi désigne soit le type List, soit le type Array}
	\\ \textbf{Utilise:}
	\\ \indent Integer, Boolean, Element \footnote{Element désigne une type générique}
	\\ \textbf{Opérations:}
	\\ \indent create\_empty: $\rightarrow$ Multi
	\\ \indent is\_empty: Multi $\rightarrow$ Boolean
	\\ \indent count: Multi $\rightarrow$ Integer
	\\ \indent occurrences: Element x Multi $\rightarrow$ Integer
	\\ \indent part\_of: Element x Multi $\rightarrow$ Boolean
	\\ \indent equals: Multi x Multi $\rightarrow$ Boolean
	\\ \indent join: Multi x Multi $\rightarrow$ Multi
	\\ \indent add\_to: Element x Multi $\rightarrow$ Multi
	\\ \indent remove\_from: Element x Multi $\rightarrow$ Multi
	\subsection{Sémantique}
	\noindent \textbf{Préconditions:}
	\\ \indent $\forall$ m $\in$ Multi, $\forall$ e $\in$ Element:
	\\ \indent \indent remove\_from(e, m) est défini ssi is\_empty(m) = False
	\\ \textbf{Axiomes:}
	\\ \indent $\forall$ m $\in$ Multi, $\forall$ e $\in$ Element:
	\\ \indent \indent is\_empty(create\_empty()) = True
	\\ \indent \indent is\_empty(add\_to(e, m)) = False
	\\ \indent \indent \underline{\textbf{Si}} is\_empty(m) \underline{\textbf{alors}} count(m) = 0
	\\ \indent \indent \underline{\textbf{sinon}} count(m) > 0
	
	\subsection{Jusitification des axiomes}

\end{document}
